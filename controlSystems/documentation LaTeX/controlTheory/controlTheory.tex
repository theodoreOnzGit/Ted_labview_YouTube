\documentclass[17pt]{extarticle}


\begin{document}

\part{Laplace Transforms}

\section{definitions}

$$\mathcal{L} \left[ f(t) \right] = F(s)$$
$$\mathcal{L} \left[ f(t) \right] = \int_0^\infty f(t) e^{-st} dt$$

\section{derivation}

\paragraph{laplace trf of 1}
$$\mathcal{L} \left[ 1 \right] = \int_0^\infty  e^{-st} dt$$

$$\mathcal{L} \left[ 1 \right] = \left[ \frac{e^{-st}}{-s} \right]_0^\infty$$

$$\mathcal{L} \left[ 1 \right] = 0 -  \frac{e^{-s*0}}{-s}$$
$$\mathcal{L} \left[ 1 \right] =  \frac{1}{s}$$

\paragraph{laplace trf of exponentials}
\subparagraph{positive exponentials}

$$\mathcal{L} \left[ e^{at} \right] = \int_0^\infty e^{at} e^{-st} dt$$

$$\mathcal{L} \left[ e^{at} \right] = \int_0^\infty  e^{(a-s)t} dt$$

$$\mathcal{L} \left[ e^{at} \right] = \left[ \frac{e^{(a-s)t}}{(a-s)}\right]_0^\infty$$

$$\mathcal{L} \left[ e^{at} \right] = \frac{e^{(a-s)\infty}}{(a-s)} - \frac{e^{(a-s)*0}}{(a-s)} $$

$$\mathcal{L} \left[ e^{at} \right] = \frac{e^{(a-s)\infty}}{(a-s)} + \frac{1}{(s-a)} $$

we have this problem:

$$\frac{e^{(a-s)\infty}}{(a-s)}$$


$$a=s \rightarrow \frac{1}{0} = \infty$$
$$a-s >0 \rightarrow \mathcal{L} \left[ e^{at} \right] = \infty$$
$$a-s <0 ; s> a ; \rightarrow \mathcal{L} \left[ e^{at} \right] = \frac{1}{s-a}$$
\subparagraph{negative exponentials}

note that a$>$0 and is real.

$$\mathcal{L} \left[ e^{-at} \right] = \int_0^\infty e^{-at} e^{-st} dt$$

$$\mathcal{L} \left[ e^{-at} \right] = \int_0^\infty  e^{-(a+s)t} dt$$

$$\mathcal{L} \left[ e^{-at} \right] = \left[ \frac{e^{-(a+s)t}}{-(a+s)}\right]_0^\infty$$

$$a+s > 0$$
$$\mathcal{L} \left[ e^{-at} \right] = \frac{e^{-(a+s)\infty}}{-(a+s)} - \frac{e^{-(a+s)*0}}{-(a+s)}$$

$$\mathcal{L} \left[ e^{-at} \right] = 0 - \frac{e^{-(a+s)*0}}{-(a+s)}$$

$$\mathcal{L} \left[ e^{-at} \right] =  \frac{1}{a+s}$$

\subparagraph{imaginary exponentials (sines and cosines)}

\begin{verbatim}
https://en.wikipedia.org/wiki/Euler%27s_formula
\end{verbatim}

$$\cos (\omega t) = \frac{\exp (i \omega t) + \exp (- i \omega t)}{2}$$
$$\sin (\omega t) = \frac{\exp (i \omega t) - \exp (- i \omega t)}{2i}$$

$$\mathcal{L} \left[ e^{-i\omega t} \right] = \int_0^\infty e^{-i\omega t} e^{-st} dt$$

$$\mathcal{L} \left[ e^{-i\omega t} \right] = \int_0^\infty e^{-(i\omega+s) t} dt$$

$$\mathcal{L} \left[ e^{-i\omega t} \right] = \left[ \frac{e^{-(i \omega+s)t}}{-(i \omega+s)}\right]_0^\infty$$

$$(i\omega +s) > 0$$

$$\mathcal{L} \left[ e^{-i\omega t} \right] = \frac{1}{(i \omega+s)}$$

$$\mathcal{L} \left[ e^{i\omega t} \right] = \int_0^\infty e^{i\omega t} e^{-st} dt$$

$$\mathcal{L} \left[ e^{i\omega t} \right] = \left[ \frac{e^{(i\omega-s)t}}{(i\omega-s)}\right]_0^\infty$$

Assume
$$i \omega  > s$$

$$\mathcal{L} \left[ e^{i\omega t} \right] = 0 + \frac{1}{s- i \omega }$$
$$\mathcal{L} \left[ e^{i\omega t} \right] = \frac{1}{s- i \omega }$$

now we can do sines and cosines:


$$\cos (\omega t) = \frac{\exp (i \omega t) + \exp (- i \omega t)}{2}$$
$$\mathcal{L}(\cos (\omega t)) =\frac{1}{2} \frac{1}{s- i \omega } + \frac{1}{2} \frac{1}{s+i \omega}$$


$$\mathcal{L}(\cos (\omega t)) =\frac{1}{2} \left[ \frac{s+i \omega}{s^2 +  \omega^2 } + \frac{s- i \omega}{s^2 +  \omega^2} \right]$$

$$\mathcal{L}(\cos (\omega t)) = \frac{s}{s^2 +  \omega^2} $$

Let's do sines:

$$\sin (\omega t) = \frac{\exp (i \omega t) - \exp (- i \omega t)}{2i}$$
$$\mathcal{L}(\sin (\omega t)) =\frac{1}{2i} \frac{1}{s- i \omega } - \frac{1}{2i} \frac{1}{s+i \omega}$$


$$\mathcal{L}(\sin (\omega t)) =\frac{1}{2i} \left[ \frac{s+i \omega}{s^2 +  \omega^2 } - \frac{s- i \omega}{s^2 +  \omega^2} \right]$$

$$\mathcal{L}(\sin (\omega t)) = \frac{\omega}{s^2 +  \omega^2} $$

\subparagraph{complex exponentials}

$$a< 0 ; \omega <0$$

$$\mathcal{L} \left[ e^{-(a+i\omega) t} \right] = \int_0^\infty e^{-(a+i\omega) t} e^{-st} dt$$


$$\mathcal{L} \left[ e^{-(a+i\omega) t} \right] = \left[ \frac{e^{-(a+i \omega+s)t}}{-(a+ i \omega+s)}\right]_0^\infty$$

$$a+i \omega  +s >0 $$

$$\mathcal{L} \left[ e^{-(a+i\omega) t} \right] = \frac{1}{(a+ i \omega+s)}$$

$$a< 0 ; \omega > 0$$

$$\mathcal{L} \left[ e^{-(a-i\omega) t} \right] = \int_0^\infty e^{-(a-i\omega) t} e^{-st} dt$$

$$a-i \omega  + s >0 $$

$$\mathcal{L} \left[ e^{-(a-i\omega) t} \right] =  \left[ \frac{e^{-(a-i \omega+s)t}}{-(a- i \omega+s)}\right]_0^\infty$$

$$\mathcal{L} \left[ e^{-(a-i\omega) t} \right] =   \frac{1}{(a- i \omega+s)}$$

So we can apply this to sines and cosines with exponential product.

$$\mathcal{L} \left[ e^{-a t}  sin (\omega t) \right] = \mathcal{L} \left[ e^{-a t}  \frac{e^{i\omega t} -e^{-i\omega t}}{2i} \right]  = \int_0^\infty e^{-a t} e^{-st} \frac{e^{i\omega t} -e^{-i\omega t}}{2i} dt$$


$$\mathcal{L} \left[ e^{-a t}  \frac{e^{i\omega t} -e^{-i\omega t}}{2i} \right]$$
$$ = \frac{1}{2i} \mathcal{L} (e^{-at+i\omega t}) -\frac{1}{2i} \mathcal{L} (e^{-at-i\omega t}) $$

$$ = \frac{1}{2i}  \frac{1}{(a- i \omega+s)} -\frac{1}{2i}  \frac{1}{(a+ i \omega+s)} $$

Note:
$$(a- i\omega +s)(a +i \omega +s) = (s^2 +a^2 + \omega^2 +s(a-i \omega) +s (a + i \omega) ) $$

$$= s^2 + a^2 + \omega^2 + 2as = (s+a)^2 + \omega^2$$
Subs back:


$$ = \frac{1}{2i}  \frac{a+i \omega + s}{(s+a)^2 + \omega^2} -\frac{1}{2i}  \frac{a- i \omega + s}{(s+a)^2 + \omega^2} $$

$$ = \frac{1}{2i}  \frac{2 i \omega}{(s+a)^2 + \omega^2} $$
$$ \mathcal{L} \left[ e^{-a t}  sin (\omega t) \right] =  \frac{ \omega}{(s+a)^2 + \omega^2} $$



\part{Appendix}

\section{Font Sizes}

\begin{verbatim}
https://www.overleaf.com/learn/latex/Questions/How_do_I_adjust_the_font_size%3F
\end{verbatim}

\end{document}