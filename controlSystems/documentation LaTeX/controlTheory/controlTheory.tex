\documentclass[17pt]{extarticle}


\begin{document}

\part{Laplace Transforms}

\section{definitions}

$$\mathcal{L} \left[ f(t) \right] = F(s)$$
$$\mathcal{L} \left[ f(t) \right] = \int_0^\infty f(t) e^{-st} dt$$

\section{derivation}

\paragraph{laplace trf of 1}
$$\mathcal{L} \left[ 1 \right] = \int_0^\infty  e^{-st} dt$$

$$\mathcal{L} \left[ 1 \right] = \left[ \frac{e^{-st}}{-s} \right]_0^\infty$$

$$\mathcal{L} \left[ 1 \right] = 0 -  \frac{e^{-s*0}}{-s}$$
$$\mathcal{L} \left[ 1 \right] =  \frac{1}{s}$$

\paragraph{laplace trf of exponentials}
\subparagraph{positive exponentials}

$$\mathcal{L} \left[ e^{at} \right] = \int_0^\infty e^{at} e^{-st} dt$$

$$\mathcal{L} \left[ e^{at} \right] = \int_0^\infty  e^{(a-s)t} dt$$

$$\mathcal{L} \left[ e^{at} \right] = \left[ \frac{e^{(a-s)t}}{(a-s)}\right]_0^\infty$$

$$\mathcal{L} \left[ e^{at} \right] = \frac{e^{(a-s)\infty}}{(a-s)} - \frac{e^{(a-s)*0}}{(a-s)} $$

$$\mathcal{L} \left[ e^{at} \right] = \frac{e^{(a-s)\infty}}{(a-s)} + \frac{1}{(s-a)} $$

we have this problem:

$$\frac{e^{(a-s)\infty}}{(a-s)}$$


$$a=s \rightarrow \frac{1}{0} = \infty$$
$$a-s >0 \rightarrow \mathcal{L} \left[ e^{at} \right] = \infty$$
$$a-s <0 ; s> a ; \rightarrow \mathcal{L} \left[ e^{at} \right] = \frac{1}{s-a}$$
\subparagraph{negative exponentials}

note that a$>$0 and is real.

$$\mathcal{L} \left[ e^{-at} \right] = \int_0^\infty e^{-at} e^{-st} dt$$

$$\mathcal{L} \left[ e^{-at} \right] = \int_0^\infty  e^{-(a+s)t} dt$$

$$\mathcal{L} \left[ e^{-at} \right] = \left[ \frac{e^{-(a+s)t}}{-(a+s)}\right]_0^\infty$$

$$a+s > 0$$
$$\mathcal{L} \left[ e^{-at} \right] = \frac{e^{-(a+s)\infty}}{-(a+s)} - \frac{e^{-(a+s)*0}}{-(a+s)}$$

$$\mathcal{L} \left[ e^{-at} \right] = 0 - \frac{e^{-(a+s)*0}}{-(a+s)}$$

$$\mathcal{L} \left[ e^{-at} \right] =  \frac{1}{a+s}$$

\subparagraph{imaginary exponentials (sines and cosines)}

\begin{verbatim}
https://en.wikipedia.org/wiki/Euler%27s_formula
\end{verbatim}

$$\cos (\omega t) = \frac{\exp (i \omega t) + \exp (- i \omega t)}{2}$$
$$\sin (\omega t) = \frac{\exp (i \omega t) - \exp (- i \omega t)}{2i}$$

$$\mathcal{L} \left[ e^{-i\omega t} \right] = \int_0^\infty e^{-i\omega t} e^{-st} dt$$

$$\mathcal{L} \left[ e^{-i\omega t} \right] = \int_0^\infty e^{-(i\omega+s) t} dt$$

$$\mathcal{L} \left[ e^{-i\omega t} \right] = \left[ \frac{e^{-(i \omega+s)t}}{-(i \omega+s)}\right]_0^\infty$$

$$(i\omega +s) > 0$$

$$\mathcal{L} \left[ e^{-i\omega t} \right] = \frac{1}{(i \omega+s)}$$

$$\mathcal{L} \left[ e^{i\omega t} \right] = \int_0^\infty e^{i\omega t} e^{-st} dt$$

$$\mathcal{L} \left[ e^{i\omega t} \right] = \left[ \frac{e^{(i\omega-s)t}}{(i\omega-s)}\right]_0^\infty$$

Assume
$$i \omega  > s$$

$$\mathcal{L} \left[ e^{i\omega t} \right] = 0 + \frac{1}{s- i \omega }$$
$$\mathcal{L} \left[ e^{i\omega t} \right] = \frac{1}{s- i \omega }$$

now we can do sines and cosines:


$$\cos (\omega t) = \frac{\exp (i \omega t) + \exp (- i \omega t)}{2}$$
$$\mathcal{L}(\cos (\omega t)) =\frac{1}{2} \frac{1}{s- i \omega } + \frac{1}{2} \frac{1}{s+i \omega}$$


$$\mathcal{L}(\cos (\omega t)) =\frac{1}{2} \left[ \frac{s+i \omega}{s^2 +  \omega^2 } + \frac{s- i \omega}{s^2 +  \omega^2} \right]$$

$$\mathcal{L}(\cos (\omega t)) = \frac{s}{s^2 +  \omega^2} $$

Let's do sines:

$$\sin (\omega t) = \frac{\exp (i \omega t) - \exp (- i \omega t)}{2i}$$
$$\mathcal{L}(\sin (\omega t)) =\frac{1}{2i} \frac{1}{s- i \omega } - \frac{1}{2i} \frac{1}{s+i \omega}$$


$$\mathcal{L}(\sin (\omega t)) =\frac{1}{2i} \left[ \frac{s+i \omega}{s^2 +  \omega^2 } - \frac{s- i \omega}{s^2 +  \omega^2} \right]$$

$$\mathcal{L}(\sin (\omega t)) = \frac{\omega}{s^2 +  \omega^2} $$

\subparagraph{complex exponentials}

$$a< 0 ; \omega <0$$

$$\mathcal{L} \left[ e^{-(a+i\omega) t} \right] = \int_0^\infty e^{-(a+i\omega) t} e^{-st} dt$$


$$\mathcal{L} \left[ e^{-(a+i\omega) t} \right] = \left[ \frac{e^{-(a+i \omega+s)t}}{-(a+ i \omega+s)}\right]_0^\infty$$

$$a+i \omega  +s >0 $$

$$\mathcal{L} \left[ e^{-(a+i\omega) t} \right] = \frac{1}{(a+ i \omega+s)}$$

$$a< 0 ; \omega > 0$$

$$\mathcal{L} \left[ e^{-(a-i\omega) t} \right] = \int_0^\infty e^{-(a-i\omega) t} e^{-st} dt$$

$$a-i \omega  + s >0 $$

$$\mathcal{L} \left[ e^{-(a-i\omega) t} \right] =  \left[ \frac{e^{-(a-i \omega+s)t}}{-(a- i \omega+s)}\right]_0^\infty$$

$$\mathcal{L} \left[ e^{-(a-i\omega) t} \right] =   \frac{1}{(a- i \omega+s)}$$

So we can apply this to sines and cosines with exponential product.

$$\mathcal{L} \left[ e^{-a t}  sin (\omega t) \right] = \mathcal{L} \left[ e^{-a t}  \frac{e^{i\omega t} -e^{-i\omega t}}{2i} \right]  = \int_0^\infty e^{-a t} e^{-st} \frac{e^{i\omega t} -e^{-i\omega t}}{2i} dt$$


$$\mathcal{L} \left[ e^{-a t}  \frac{e^{i\omega t} -e^{-i\omega t}}{2i} \right]$$
$$ = \frac{1}{2i} \mathcal{L} (e^{-at+i\omega t}) -\frac{1}{2i} \mathcal{L} (e^{-at-i\omega t}) $$

$$ = \frac{1}{2i}  \frac{1}{(a- i \omega+s)} -\frac{1}{2i}  \frac{1}{(a+ i \omega+s)} $$

Note:
$$(a- i\omega +s)(a +i \omega +s) = (s^2 +a^2 + \omega^2 +s(a-i \omega) +s (a + i \omega) ) $$

$$= s^2 + a^2 + \omega^2 + 2as = (s+a)^2 + \omega^2$$
Subs back:


$$ = \frac{1}{2i}  \frac{a+i \omega + s}{(s+a)^2 + \omega^2} -\frac{1}{2i}  \frac{a- i \omega + s}{(s+a)^2 + \omega^2} $$

$$ = \frac{1}{2i}  \frac{2 i \omega}{(s+a)^2 + \omega^2} $$
$$ \mathcal{L} \left[ e^{-a t}  sin (\omega t) \right] =  \frac{ \omega}{(s+a)^2 + \omega^2} $$

\paragraph{Derivatives and integrals}

$$\mathcal{L} [ \frac{df(t)}{dt}] = \int_0^\infty \frac{d f(t)}{dt} \exp(-st) dt $$

Integrate by parts:

$$\mathcal{L} [ \frac{df(t)}{dt}] = [f(t) \exp(-st)]_0^\infty - \int_0^\infty f(t) (-s) \exp(-st) dt$$

$$\mathcal{L} [ \frac{df(t)}{dt}] = [f(t) \exp(-st)]_0^\infty +s \int_0^\infty f(t)  \exp(-st) dt$$

$$\mathcal{L} [ \frac{df(t)}{dt}] = [f(t) \exp(-st)]_0^\infty +s \mathcal{L} (f(t))$$

$$\mathcal{L} [ \frac{df(t)}{dt}] = [f(t \rightarrow \infty) \exp(-s\infty) - f(t = 0) * \exp(-s*0) ] +s \mathcal{L} (f(t))$$

$$f(t) \neq \exp(st)$$
and $f(t)$ doesn't go to infinity faster than $\exp(-st)$, s$>$0

$$\mathcal{L} [ \frac{df(t)}{dt}] = [0 - f(t = 0) ] +s \mathcal{L} (f(t))$$

$$\mathcal{L} [ \frac{df(t)}{dt}] = s \mathcal{L} (f(t)) - f(t = 0) $$

We can use this transform for integrals as long as u have the BC

$$\mathcal{L} [ f(t)] = s \mathcal{L} (\int f(t) dt) -\int f (t) dt|_{t = 0} )  $$

$$ \mathcal{L} (\int f(t) dt) = \frac{1}{s} \left[ [\mathcal{L} [ f(t)] + \int f (t) dt|_{t = 0} ]  \right]$$


We can see in laplace domain, integrating means multiply by $\frac{1}{s}$ and differentiating is multiplying by s. Of course we have BCs.

Let's do some examples,

$$\mathcal{L}[t]$$

Note:

$$\frac{d t}{dt} =1 $$

We can use this:

$$ \mathcal{L} (\int f(t) dt) = \frac{1}{s} \left[ [\mathcal{L} [ f(t)] + \int f (t) dt|_{t = 0} ]  \right]$$

in this case $f(t)=1$

$$ \mathcal{L} (t) = \frac{1}{s} \left[ [\mathcal{L} [1] + \int 1 dt|_{t = 0} ]  \right]$$

$$ \mathcal{L} (t) = \frac{1}{s} \left[ [\mathcal{L} [1] + t|_{t = 0} ]  \right]$$

$$ \mathcal{L} (t) = \frac{1}{s} \left[ [\mathcal{L} [1] + 0 ]  \right]$$

$$ \mathcal{L} (t) = \frac{1}{s} \left[ [\frac{1}{s} ]  \right] = \frac{1}{s^2}$$


Let's do 
$$\mathcal{L}[t^2]$$


Let's use the integration formula, say f(t) $=2t$

$$ \mathcal{L} (\int f(t) dt) = \frac{1}{s} \left[ [\mathcal{L} [ f(t)] + \int f (t) dt|_{t = 0} ]  \right]$$


$$ \mathcal{L} (t^2) = \frac{1}{s} \left[ [2\mathcal{L} [ t]   + t^2_{t = 0} ]  \right]$$

$$ \mathcal{L} (t^2) = \frac{1}{s}  [2 \frac{1}{s^2} ] = \frac{2}{s^3}  $$

\part{applications examples}

\section{ODEs of time varying systems}

linear time invariant (LTI) systems
$$y'' + 2y' + y = \sin (\omega t)$$
$$y'+3y = \exp(-{2t}) + 4$$

nonlinear:

$$(y'')^2 + 2y' + y = \sin (\omega t)$$

2nd order ODE:

$$\mathcal{L}(y''+2y'+y)= \mathcal{L}(\sin (\omega t)$$


$$\mathcal{L}(y''+2y'+y)= \frac{\omega}{s^2 + \omega^2}$$

$$\mathcal{L}(y'') + y(s) + 2[s y(s) - y(t=0)]= \frac{\omega}{s^2 + \omega^2}$$

$$\mathcal{L}(y'') = s \mathcal{L}(y'(t)) - y'(t=0)$$

$$\mathcal{L}(y'') = s (s y(s) - y(t=0)) - y'(t=0)$$
$$\mathcal{L}(y'') =  s^2 y(s) - sy(t=0) - y'(t=0)$$

subs back:

$$s^2 y(s) - sy(t=0) - y'(t=0)+ y(s) + 2[s y(s) - y(t=0)]= \frac{\omega}{s^2 + \omega^2}$$


$$s^2 y(s) + y(s) + [2s y(s) - 2y(t=0)]= \frac{\omega}{s^2 + \omega^2} + sy(t=0) + y'(t=0)$$

$$s^2 y(s) + y(s) + 2s y(s) = \frac{\omega}{s^2 + \omega^2} + sy(t=0) + y'(t=0) + 2y(t=0)$$

$$s^2 y(s) + y(s) + 2s y(s) = \frac{\omega}{s^2 + \omega^2} + (2+s) y(t=0) + y'(t=0) $$

$$y (s) (s^2 + 2s + 1)= \frac{\omega}{s^2 + \omega^2} + (2+s) y(t=0) + y'(t=0) $$

$$y (s)= \frac{\omega}{s^2 + \omega^2} \frac{1}{ (s^2 + 2s + 1)} +  \frac{(2+s) y(t=0) + y'(t=0)}{s^2 + 2s + 1} $$

$$y (s)= \frac{\omega}{s^2 + \omega^2} \frac{1}{ (s^2 + 2s + 1)} +  \frac{(2+s) y(t=0) }{s^2 + 2s + 1} + \frac{y'(t=0)}{s^2 + 2s + 1} $$


$$y(t)  = \mathcal{L}^{-1} (\frac{\omega}{s^2 + \omega^2} \frac{1}{ (s^2 + 2s + 1)} +  \frac{(2+s) y(t=0) }{s^2 + 2s + 1} + \frac{y'(t=0)}{s^2 + 2s + 1} )$$


$$y(t)  = \mathcal{L}^{-1} (\frac{\omega}{s^2 + \omega^2} \frac{1}{(s+1)^2} +  \frac{(2+s) y(t=0) }{(s+1)^2} + \frac{y'(t=0)}{(s+1)^2} )$$

Note:

$$\mathcal{L}[e^{-t}] = \frac{1}{s+1}$$

Partial fraction:

$$\frac{\omega}{s^2 + \omega^2} \frac{1}{(s+1)^2}  = \frac{Bs + C}{s^2 + \omega^2} + \frac{D}{s+1} + \frac{E}{(s+1)^2} $$

B, C, D and E are constants


$$\frac{\omega}{s^2 + \omega^2} \frac{1}{(s+1)^2}  = \frac{Bs + C}{s^2 + \omega^2} + \frac{D}{s+1} + \frac{E}{(s+1)^2} $$

\begin{verbatim}
https://www.wolframalpha.com/input/?i=partial+fraction+1%2F%28%28x%2Ba%29%5E2+*+%28x%5E2%2Bb%5E2%29%29
\end{verbatim}

From wolfram:
$$\frac{1}{s^2 + \omega^2} \frac{1}{(s+1)^2} = \frac{-\omega^2 - 2s +1}{(\omega^2+1)^2} \frac{1}{\omega^2 + s^2}+ \frac{2}{(\omega^2+1)^2} \frac{1}{s+1} + \frac{1}{\omega^2+1} \frac{1}{(s+1)^2}$$

multiply by $\omega$

$$\frac{\omega}{s^2 + \omega^2} \frac{1}{(s+1)^2} = \frac{-\omega^2 - 2s +1}{(\omega^2+1)^2} \frac{\omega}{\omega^2 + s^2}+ \frac{2\omega}{(\omega^2+1)^2} \frac{1}{s+1} + \frac{\omega}{\omega^2+1} \frac{1}{(s+1)^2}$$


$$E = \frac{\omega}{\omega^2+1}$$
$$D = \frac{2\omega}{(\omega^2+1)^2}$$
$$B = \frac{-2\omega}{(\omega^2+1)^2}$$
$$C = \frac{\omega(1-\omega^2)}{(\omega^2+1)^2}$$

So we can reduce the first term to this:

$$\frac{\omega}{s^2 + \omega^2} \frac{1}{(s+1)^2}  = \frac{Bs + C}{s^2 + \omega^2} + \frac{D}{s+1} + \frac{E}{(s+1)^2} $$


How do we inverse laplace $\frac{1}{(s+1)^2}$?

\subparagraph{Frequency Shift}

$$\mathcal{L}[e^{-at} f(t)] = \int_0^\infty e^{-at} f(t) e^{-st} dt $$

$$\mathcal{L}[e^{-at} f(t)] = \int_0^\infty f(t) e^{-(s+a)t} dt $$

Compare this with:

$$\mathcal{L}[f(t)] = \int_0^\infty f(t) e^{-st} dt = F(s)$$

So we can say:

$$\mathcal{L}[e^{-at} f(t)] = \int_0^\infty f(t) e^{-(s+a)t} dt = F(s+a) $$

let's say

$$\mathcal{L}(t^2)= \frac{2}{s^3}$$

$$\mathcal{L}(t^2 \exp(-at))= \frac{2}{(s+a)^3}$$
\subparagraph{back to the qn}
Now back to: $\frac{1}{(s+1)^2}$

$$\mathcal{L}(t)= \frac{1}{s^2}$$

$$\mathcal{L}(te^{-at})= \frac{1}{(s+1)^2}$$
$$te^{-at}= \mathcal{L}^{-1} \left[ \frac{1}{(s+1)^2} \right]$$

back to main qn:

$$y(t)  = \mathcal{L}^{-1} (\frac{\omega}{s^2 + \omega^2} \frac{1}{(s+1)^2} +  \frac{(2+s) y(t=0) }{(s+1)^2} + \frac{y'(t=0)}{(s+1)^2} )$$

$$\frac{2+s}{(s+1)^2} = \frac{s+1 +1}{(s+1)^2}= \frac{1}{s+1} + \frac{1}{(s+1)^2}$$

$$y(t)  = \mathcal{L}^{-1} (\frac{\omega}{s^2 + \omega^2} \frac{1}{(s+1)^2} +  \frac{(2+s) y(t=0) }{(s+1)^2} + \frac{y'(t=0)}{(s+1)^2} )$$

$$= \mathcal{L}^{-1} \left(\frac{\omega}{s^2 + \omega^2} \frac{1}{(s+1)^2} + y(t=0)\left[ \frac{1  }{(s+1)^2}  + \frac{1}{s+1} \right] + \frac{y'(t=0)}{(s+1)^2} \right)$$


$$= \mathcal{L}^{-1} \left(\frac{\omega}{s^2 + \omega^2} \frac{1}{(s+1)^2}\right) + y(t=0)\left[ te^{-t}  + e^{-t} \right] + y'(t=0) te^{-t} $$


$$= \mathcal{L}^{-1}(\frac{Bs}{(s^2+\omega^2)}+ \frac{C}{(s^2+\omega^2)}+ \frac{D}{s+1}+ \frac{E}{(s+1)^2}) + y(t=0)\left[ te^{-t} + e^{-t} \right] 
$$
$$+ y'(t=0) te^{-t} $$

$$= D e^{-t} + E te^{-t} + y(t=0)\left[ te^{-t} + e^{-t} \right] 
$$
$$+ y'(t=0) te^{-t} +B \cos (\omega t) + \frac{C}{\omega} \sin (\omega t)$$

$$E = \frac{\omega}{\omega^2+1}$$
$$D = \frac{2\omega}{(\omega^2+1)^2}$$
$$B = \frac{-2\omega}{(\omega^2+1)^2}$$
$$C = \frac{\omega(1-\omega^2)}{(\omega^2+1)^2}$$

Leaving it as it is...

Example 2: first order ODE

$$y' +3y = \exp(-2t) + 4$$

$$s Y(s) - y(t=0) + 3 Y(s) = \frac{1}{s+2} + \frac{4}{s}$$

$$s Y(s) + 3 Y(s) = \frac{1}{s+2} + \frac{4}{s} +  y(t=0)$$

$$Y(s)  (s + 3 ) = \frac{1}{s+2} + \frac{4}{s} +  y(t=0)$$
$$Y(s)   = \frac{1}{s+2} \frac{1}{(s + 3 )} + \frac{4}{s} \frac{1}{(s + 3 )} +  y(t=0) \frac{1}{s+3}$$

$$Y(s)   = \frac{1}{s+2} \frac{1}{(s + 3 )} + \frac{4}{s} \frac{1}{(s + 3 )} +  y(t=0) \frac{1}{s+3}$$

$$y(t)   = \mathcal{L}^{-1} \left( \frac{1}{s+2} \frac{1}{(s + 3 )} + \frac{4}{s} \frac{1}{(s + 3 )} +  y(t=0) \frac{1}{s+3} \right)$$


$$y(t)   = \mathcal{L}^{-1} \left( \frac{1}{s+2} \frac{1}{(s + 3 )} + \frac{4}{s} \frac{1}{(s + 3 )}  \right)+ y(t=0) \exp(-3t)$$

wolfram partial fraction

$$\frac{1}{s+2} \frac{1}{(s + 3 )} = \frac{1}{s+2} - \frac{1}{s+3}$$


note: careless mistake in video, i forgot to times 4
$$y(t)   = \mathcal{L}^{-1} \left( \frac{1}{s+2} - \frac{1}{s+3} + \frac{4}{s} \frac{1}{(s + 3 )}  \right)+ y(t=0) \exp(-3t)$$

$$y(t)   = \mathcal{L}^{-1} \left( \frac{4}{s} \frac{1}{(s + 3 )}  \right)+ y(t=0) \exp(-3t) + \exp(-2t) - \exp(-3t)$$

$$y(t)   = \mathcal{L}^{-1} \left( \frac{4/3}{s} -  \frac{1/3}{s + 3 }  \right)+ y(t=0) \exp(-3t) + \exp(-2t) - \exp(-3t)$$

$$y(t)   =  \frac{4}{3} - \frac{1}{3} e^{-3t}+ y(t=0) \exp(-3t) + \exp(-2t) - \exp(-3t)$$

$$y(t)   =  \frac{4}{3} - \frac{1}{3} e^{-3t}+ [y(t=0)-1] \exp(-3t) + \exp(-2t) $$

$$y(t)   =  \frac{4}{3} + \left[y(t=0)-\frac{4}{3} \right] \exp(-3t) + \exp(-2t) $$

\part{Appendix}

\section{Font Sizes}

\begin{verbatim}
https://www.overleaf.com/learn/latex/Questions/How_do_I_adjust_the_font_size%3F
\end{verbatim}

\end{document}